\documentclass[11pt]{article}
\usepackage{latexsym}
\usepackage{listings}
\setlength{\parindent}{0mm}
\setlength{\parskip}{1ex}
\textwidth=15.7cm
\textheight=22.9cm
\voffset=-2.54cm
\hoffset=-2.7cm

\begin{document}
Description of the ``infinite population'' version of the discrete model of the 
``Cultural evolution is rare'' paper of Boyd Richerson (1996).

Model parameters:

$W_0 = $ base fitness \\
$\delta = $ probability of individual learner (IL) learning skill\\
$C_\ell = $ fitness cost of any individual attempting to learn skill individually\\
$C_s = $ fitness costs of social learner (SL) learning skill (no penalty for attempt)\\
$K = $ fitness penalty for being a social learner\\
$D = $ fitness bonus for knowing skill\\
$n = $ number of previous generation individuals observed by SL\\
$\gamma = $ per generation probability of environment change\\

Simulation parameters:

$G = $ number of generations\\

Derived quantities:

$p = $ probability of a previous generation SL being skilled\\
$q = $ probability of a previous generation individual being skilled\\
$1-(1-q)^n = $ probability of SL learning skill assuming no environment change\\
$W_\ell = W_0 + \delta D - C_\ell = $ expected fitness of IL \\
$W_s = \gamma(W_0+\delta D - C_\ell) + (1-\gamma)(W_0+\pi(D-C_s)+(1-\pi)(\delta D-C_\ell))
 = $ expected fitness of SL \\

Infinite population model assuming no environment change:

Keep track of:  

$s_t = $ frequency of social learners in generation $t$\\
$q_t = $ frequency of skilled individuals in generation $t$\\

Note:  Frequency means relative frequency (out of a total of 1.0).
Thus, $1-s_t = $ frequency of individual learners.

Top-level algorithm:

\begin{tabbing}
hhh\=hhh\=hhh\=hhh\=hhh\kill
\textsc{ BoydRichersonInfPop}($G$,$s_1$)\\
\>for $t$ from 1 to $G$:\\
\>\>learning: produces freq of skilled individual learners and skilled social learners\\
\>\>proportional selection: produces $s_{t+1}$\\
\end{tabbing}


%\clearpage
Outcomes of learning and their fitnesses:

\begin{tabular}{|l|l|l|}
\hline
Outcome & Probability & Fitness\\
\hline
IL unskilled & $ (1-s_t)(1-\delta) $ & $W_0 - C_\ell$\\
IL skilled & $ (1-s_t)\delta $ & $W_0 + D - C_\ell$\\
SL social learned skilled & $s_t p_t$ & $W_0 + D - C_s - K$\\
SL indiv learned skilled & $s_t (1-p_t)\delta$ & $W_0 + D - C_\ell - K$\\
SL unskilled & $s_t (1-p_t)(1-\delta)$ & $W_0 - C_\ell - K$\\
\hline
\end{tabular}

where
$p_t = 1-(1-q_{t-1})^n = $ probability of an SL individual learning skill in generation $t$

We can calculate $q_t$ by:\\
$q_t = (1-s_t)\delta + s_t( p_t + (1-p_t)\delta) = \delta + s_t p_t (1-\delta)$\\
\hspace{0.2in} = probability of an individual being skilled in generation $t$\\


\begin{tabbing}
hhh\=hhh\=hhh\=hhh\=hhh\kill
\textsc{ BoydRichersonInfPop}($G$,$s_1$)\\
\>$q_0 \leftarrow 0$\\
\>for $t$ from 1 to $G$:\\
\>\> $p_t \leftarrow 1-(1-q_{t-1})^n$\\ 
\>\> $q_t \leftarrow \delta + s_t p_t (1-\delta)$\\
\>\> $indiv \leftarrow (1-s_t)(1-\delta)(W_0-C_\ell) +(1-s_t)\delta (W_0 + D - C_\ell)$\\
\>\> $social \leftarrow s_t p_t (W_0 + D - C_s - K) +s_t (1-p_t)\delta (W_0 + D - C_\ell - K) 
      +s_t (1-p_t)(1-\delta)(W_0-C_\ell-K)$\\
\>\> $total \leftarrow indiv + social$ \\
\>\> $s_{t+1} = social/total$\\
\end{tabbing}


\end{document}
